\documentclass[ignorenonframetext,]{beamer}
\setbeamertemplate{caption}[numbered]
\setbeamertemplate{caption label separator}{:}
\setbeamercolor{caption name}{fg=normal text.fg}
\usepackage{amssymb,amsmath}
\usepackage{ifxetex,ifluatex}
\usepackage{fixltx2e} % provides \textsubscript
\usepackage{lmodern}
\ifxetex
  \usepackage{fontspec,xltxtra,xunicode}
  \defaultfontfeatures{Mapping=tex-text,Scale=MatchLowercase}
  \newcommand{\euro}{€}
\else
  \ifluatex
    \usepackage{fontspec}
    \defaultfontfeatures{Mapping=tex-text,Scale=MatchLowercase}
    \newcommand{\euro}{€}
  \else
    \usepackage[T1]{fontenc}
    \usepackage[utf8]{inputenc}
      \fi
\fi
% use upquote if available, for straight quotes in verbatim environments
\IfFileExists{upquote.sty}{\usepackage{upquote}}{}
% use microtype if available
\IfFileExists{microtype.sty}{\usepackage{microtype}}{}
\usepackage{color}
\usepackage{fancyvrb}
\newcommand{\VerbBar}{|}
\newcommand{\VERB}{\Verb[commandchars=\\\{\}]}
\DefineVerbatimEnvironment{Highlighting}{Verbatim}{commandchars=\\\{\}}
% Add ',fontsize=\small' for more characters per line
\usepackage{framed}
\definecolor{shadecolor}{RGB}{248,248,248}
\newenvironment{Shaded}{\begin{snugshade}}{\end{snugshade}}
\newcommand{\KeywordTok}[1]{\textcolor[rgb]{0.13,0.29,0.53}{\textbf{{#1}}}}
\newcommand{\DataTypeTok}[1]{\textcolor[rgb]{0.13,0.29,0.53}{{#1}}}
\newcommand{\DecValTok}[1]{\textcolor[rgb]{0.00,0.00,0.81}{{#1}}}
\newcommand{\BaseNTok}[1]{\textcolor[rgb]{0.00,0.00,0.81}{{#1}}}
\newcommand{\FloatTok}[1]{\textcolor[rgb]{0.00,0.00,0.81}{{#1}}}
\newcommand{\CharTok}[1]{\textcolor[rgb]{0.31,0.60,0.02}{{#1}}}
\newcommand{\StringTok}[1]{\textcolor[rgb]{0.31,0.60,0.02}{{#1}}}
\newcommand{\CommentTok}[1]{\textcolor[rgb]{0.56,0.35,0.01}{\textit{{#1}}}}
\newcommand{\OtherTok}[1]{\textcolor[rgb]{0.56,0.35,0.01}{{#1}}}
\newcommand{\AlertTok}[1]{\textcolor[rgb]{0.94,0.16,0.16}{{#1}}}
\newcommand{\FunctionTok}[1]{\textcolor[rgb]{0.00,0.00,0.00}{{#1}}}
\newcommand{\RegionMarkerTok}[1]{{#1}}
\newcommand{\ErrorTok}[1]{\textbf{{#1}}}
\newcommand{\NormalTok}[1]{{#1}}
\usepackage{url}
\usepackage{graphicx}
\makeatletter
\def\maxwidth{\ifdim\Gin@nat@width>\linewidth\linewidth\else\Gin@nat@width\fi}
\def\maxheight{\ifdim\Gin@nat@height>\textheight0.8\textheight\else\Gin@nat@height\fi}
\makeatother
% Scale images if necessary, so that they will not overflow the page
% margins by default, and it is still possible to overwrite the defaults
% using explicit options in \includegraphics[width, height, ...]{}
\setkeys{Gin}{width=\maxwidth,height=\maxheight,keepaspectratio}

% Comment these out if you don't want a slide with just the
% part/section/subsection/subsubsection title:
\AtBeginPart{
  \let\insertpartnumber\relax
  \let\partname\relax
  \frame{\partpage}
}
\AtBeginSection{
  \let\insertsectionnumber\relax
  \let\sectionname\relax
  \frame{\sectionpage}
}
\AtBeginSubsection{
  \let\insertsubsectionnumber\relax
  \let\subsectionname\relax
  \frame{\subsectionpage}
}

\setlength{\parindent}{0pt}
\setlength{\parskip}{6pt plus 2pt minus 1pt}
\setlength{\emergencystretch}{3em}  % prevent overfull lines
\setcounter{secnumdepth}{0}
\usepackage{url}

\title{Crash Introduction to markovchain R package}
\author{Giorgio Alfredo Spedicato, Ph.D C.Stat ACAS}
\date{19th April 2015}

\begin{document}
\frame{\titlepage}

\begin{frame}
\tableofcontents[hideallsubsections]
\end{frame}

\begin{frame}

\end{frame}

\begin{frame}{Intro}

\begin{itemize}[<+->]
\itemsep1pt\parskip0pt\parsep0pt
\item
  The markovchain package (Spedicato 2015) will be introduced.
\item
  The package is intended to provide S4 classes to perform probabilistic
  and statistical analysis of Discrete Time Markov Chains (DTMC). See
  (Br{é}maud 1999) for a theoretical review of the mathematics
  underlying the DTMC models.
\item
  The vignette will show: how to load the package and create a DTMC, how
  to manage a DTMC, how to perform basic probabilistic analysis, how to
  fit a DTMC.
\end{itemize}

\end{frame}

\begin{frame}

\begin{itemize}[<+->]
\itemsep1pt\parskip0pt\parsep0pt
\item
  The package is on Cran since Summer 2013.
\item
  It requires a recent version of R (\textgreater{}=3.0). Since version
  0.2 parts of code have been moved to Rcpp (Eddelbuettel 2013).
\item
  The package won a slot in Google Summer of Code 2015 for optimizing
  internals and expanding functionalities.
\end{itemize}

\end{frame}

\begin{frame}[fragile]{First moves into the markovchain package}

\begin{block}{Loading the package}

\begin{itemize}[<+->]
\itemsep1pt\parskip0pt\parsep0pt
\item
  The package is loaded using
\end{itemize}

\begin{Shaded}
\begin{Highlighting}[]
\KeywordTok{library}\NormalTok{(markovchain) }\CommentTok{#load the package}
\end{Highlighting}
\end{Shaded}

\end{block}

\end{frame}

\begin{frame}[fragile]

\begin{block}{Creating a DTMC}

\begin{itemize}[<+->]
\itemsep1pt\parskip0pt\parsep0pt
\item
  DTMC can be easily create following standard S4 classes syntax. The
  show method displays it.
\end{itemize}

\begin{Shaded}
\begin{Highlighting}[]
\NormalTok{tmA <-}\StringTok{ }\KeywordTok{matrix}\NormalTok{(}\KeywordTok{c}\NormalTok{(}\DecValTok{0}\NormalTok{,}\FloatTok{0.5}\NormalTok{,}\FloatTok{0.5}\NormalTok{,.}\DecValTok{5}\NormalTok{,}\DecValTok{0}\NormalTok{,.}\DecValTok{5}\NormalTok{,.}\DecValTok{5}\NormalTok{,.}\DecValTok{5}\NormalTok{,}\DecValTok{0}\NormalTok{),}\DataTypeTok{nrow =} \DecValTok{3}\NormalTok{,}\DataTypeTok{byrow =} \OtherTok{TRUE}\NormalTok{) }\CommentTok{#define the transition matrix}
\NormalTok{dtmcA <-}\StringTok{ }\KeywordTok{new}\NormalTok{(}\StringTok{"markovchain"}\NormalTok{,}\DataTypeTok{transitionMatrix=}\NormalTok{tmA, }\DataTypeTok{states=}\KeywordTok{c}\NormalTok{(}\StringTok{"a"}\NormalTok{,}\StringTok{"b"}\NormalTok{,}\StringTok{"c"}\NormalTok{), }\DataTypeTok{name=}\StringTok{"MarkovChain A"}\NormalTok{) }\CommentTok{#create the DTMC}
\NormalTok{dtmcA}
\end{Highlighting}
\end{Shaded}

\begin{verbatim}
## MarkovChain A 
##  A  3 - dimensional discrete Markov Chain with following states 
##  a b c 
##  The transition matrix   (by rows)  is defined as follows 
##     a   b   c
## a 0.0 0.5 0.5
## b 0.5 0.0 0.5
## c 0.5 0.5 0.0
\end{verbatim}

\begin{itemize}[<+->]
\itemsep1pt\parskip0pt\parsep0pt
\item
  Otherwise, it can also be created directly coercing a matrix.
\end{itemize}

\begin{Shaded}
\begin{Highlighting}[]
\NormalTok{dtmcA2<-}\KeywordTok{as}\NormalTok{(tmA, }\StringTok{"markovchain"}\NormalTok{) }\CommentTok{#using coerce from matrix}
\KeywordTok{states}\NormalTok{(dtmcA2) }\CommentTok{#note default names assigned to states}
\end{Highlighting}
\end{Shaded}

\begin{verbatim}
## [1] "s1" "s2" "s3"
\end{verbatim}

\end{block}

\end{frame}

\begin{frame}[fragile]

\begin{itemize}[<+->]
\itemsep1pt\parskip0pt\parsep0pt
\item
  It is also possible to display a DTMC, using (Csardi and Nepusz 2006)
  capabilities
\end{itemize}

\begin{Shaded}
\begin{Highlighting}[]
\KeywordTok{plot}\NormalTok{(dtmcA)}
\end{Highlighting}
\end{Shaded}

\includegraphics{markovchainCrashIntro_files/figure-beamer/plot-1.pdf}

\end{frame}

\begin{frame}{Probabilistic analysis}

\end{frame}

\begin{frame}[fragile]{The basic}

\begin{itemize}[<+->]
\itemsep1pt\parskip0pt\parsep0pt
\item
  It is possible to access transition probabilities and to perform basic
  operations.
\item
  Similarly, it is possible to access the conditional distribution of
  states, \(Pr\left ( X_{t+1} | X_{t}=s \right )\)
\end{itemize}

\begin{Shaded}
\begin{Highlighting}[]
\NormalTok{dtmcA[}\DecValTok{2}\NormalTok{,}\DecValTok{3}\NormalTok{] }\CommentTok{#using [ method}
\end{Highlighting}
\end{Shaded}

\begin{verbatim}
## [1] 0.5
\end{verbatim}

\begin{Shaded}
\begin{Highlighting}[]
\KeywordTok{transitionProbability}\NormalTok{(dtmcA, }\StringTok{"b"}\NormalTok{,}\StringTok{"c"}\NormalTok{) }\CommentTok{#using specific S4 method}
\end{Highlighting}
\end{Shaded}

\begin{verbatim}
## [1] 0.5
\end{verbatim}

\begin{Shaded}
\begin{Highlighting}[]
\KeywordTok{conditionalDistribution}\NormalTok{(dtmcA,}\StringTok{"b"}\NormalTok{)}
\end{Highlighting}
\end{Shaded}

\begin{verbatim}
##   a   b   c 
## 0.5 0.0 0.5
\end{verbatim}

\end{frame}

\begin{frame}[fragile]

\begin{itemize}[<+->]
\itemsep1pt\parskip0pt\parsep0pt
\item
  It is possible to simulate states distribution after n-steps
\end{itemize}

\begin{verbatim}
##           a     b      c
## [1,] 0.3125 0.375 0.3125
\end{verbatim}

\end{frame}

\begin{frame}[fragile]

\begin{itemize}[<+->]
\itemsep1pt\parskip0pt\parsep0pt
\item
  As well as steady states distribution
\end{itemize}

\begin{verbatim}
##              a         b         c
## [1,] 0.3333333 0.3333333 0.3333333
\end{verbatim}

\begin{itemize}[<+->]
\itemsep1pt\parskip0pt\parsep0pt
\item
  The summary method shows the proprieties of the DTCM
\end{itemize}

\begin{Shaded}
\begin{Highlighting}[]
\KeywordTok{summary}\NormalTok{(mcMathematica)}
\end{Highlighting}
\end{Shaded}

\begin{verbatim}
## Mathematica  Markov chain that is composed by: 
## Closed classes: 
## a b c d 
## Transient classes: 
## NONE 
## The Markov chain is irreducible 
## The absorbing states are: NONE
\end{verbatim}

\end{frame}

\begin{frame}[allowframebreaks]{Bibliography}

Br{é}maud, Pierre. 1999. ``Discrete-Time Markov Models.'' In
\emph{Markov Chains}, 53--93. Springer.

Csardi, Gabor, and Tamas Nepusz. 2006. ``The Igraph Software Package for
Complex Network Research.'' \emph{InterJournal} Complex Systems: 1695.
\url{http://igraph.sf.net}.

Eddelbuettel, Dirk. 2013. \emph{Seamless R and C++ Integration with
Rcpp}. New York: Springer-Verlag.

Spedicato, Giorgio Alfredo. 2015. \emph{Markovchain: An R Package to
Easily Handle Discrete Markov Chains}.

\end{frame}

\end{document}
